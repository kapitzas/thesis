\chapter{Toward streamlined and transparent economic-ecological predictions of land use change impacts on biodiversity}\label{ch4}
\newpage

\section{Abstract}

Climate and land-use change are two of the most impactful drivers of global biodiversity change, but integrated assessments of the impacts of these drivers on biodiversity are rare and tend to be highly complex and not repeatable by analysts other than those who developed them. We have developed a dramatically simplified, more transparent, and transportable integrated biodiversity assessment framework and modelling apparatus that can be modified and adapted to a broad array of integrated biodiversity assessment contexts, applicable to a range of fundamental questions about the combined impacts of economy, climate change and land-use change on biodiversity. Here, we address how closely our framework approximates more complex, more highly parameterized and less transparent and repeatable integrated assessment modelling (IAM) frameworks in terms of predictions about biodiversity change under future socio economic scenarios. We utilize our framework to predict future global biodiversity intactness under two so-called “shared socio-economic pathways” (SSPs) and compare our predictions to those derived from the “Message 8.5” IAM. We find predicted changes of biodiversity intactness were smaller when using our simplified framework compared with predictions derived from the Message 8.5 IAM.  Differences in predicted biodiversity intactness were mainly due to differences in the predicted availability of ‘natural’ habitats, that were predicted to be more extensive under our framework because our framework predicts habitat change as a function of mean land-use suitability rather than forestry and vegetation dynamics. Message 8.5 predictions are likely more close approximations of future changes in natural habitats, because of the comparatively more highly parametrized component predicting this class. Despite this caveat, we are able to showcase the first global implementation of a simple modelling framework that is more transparent, interpretable and reproducible than most other IAM.

\section{Introduction}

\subsection{Multiple threats to future biodiversity}

Biodiversity change is driven by the direct exploitation of organisms, climate change, land/sea-use change, pollution and the invasion of alien species \citep{ipbes_summary_2019}.

Climate change and land-use change are of particular significance for biodiversity and are likely to cause significant distributional changes, reduction in biodiversity and further extinctions of species \citep{ipbes_summary_2019, struebig_targeted_2015, newbold2019climate, kapitza_assessing_2021, foley_global_2005}. Climate change affects the distributional ranges of species directly by impacting on their biophysical niches and indirectly by altering global land-use patterns \citep{kapitza_assessing_2021}. 

Despite the importance of these drivers, land-use and climate change impacts on biodiversity are usually considered in isolation and only few examples exist in which synergistic impacts are examined, probably leading to the underestimation human impacts on biodiversity \citep{de_chazal_land-use_2009}. Interactions between climate and land-use on biodiversity also depend on the regional context and global economic processes that may play out across continents. For example, direct climate change impacts on biodiversity have been observed to overshadow indirect impacts (where climate change affects species distributions via impacts on land use \citep[\Cref{ch1},][]{kapitza_assessing_2021}. However, the opposite may occur if global tele-coupling of consumption and production patterns displace impacts on biodiversity to other locations. Such indirect impacts of climate change on biodiversity cannot be understood, assessed or predicted without explicitly consider economic trade. Regions where land becomes less productive under climate change may increase imports of agricultural commodities and export impacts of increasing agricultural land demands to other regions \citep{kapitza_assessing_2021, chaudhary_land_2016}. To alleviate prediction uncertainty it is critical to deepen our knowledge of when and where land-use changes and associated biodiversity impacts may occur in response to macro-economic dynamics; not only under climate change, but also under other socio-economic changes.

\subsection{Integrated biodiversity assessments}
It has been broadly acknowledged that assessments of future biodiversity require integrated, flexible models that deploy methods from various disciplines \citep{ipbes_summary_2019}, but frameworks combining such methods specifically with biodiversity in mind are still lacking \citep{titeux_global_2017}. Existing institutional-scale, integrated assessment models (IAM) have been widely adopted to inform climate change mitigation policy, although in that capacity they have tended to focus on the assessment of socio-economic and other non-human environmental factors, such as population growth, greenhouse-gas emissions, radiative forcing, land-use change and climate mitigation potential \citep{riahi_shared_2017}. Despite the high level of institutional support and technical finesse, IAMs were not designed to deliver fine-tuned predictions of biodiversity impacts.

However, this situation is slowly changing, as research concerned with predicting future environmental conditions to inform policy-making is becoming increasingly interdisciplinary. Bespoke frameworks that cross disciplinary boundaries have been developed for integrated assessments of biodiversity change \citep{newbold2019climate, kapitza_assessing_2021, leclere_bending_2020} and IAM have recently been deployed to predict regional biodiversity outcomes to inform sustainable development policy \citep{veerkamp_future_2020}, confirming that biodiversity is increasingly being acknowledged as a crucial aspect of sustainable policy-making.

\subsection{Shared future narratives}
Efforts to harmonize future predictions made by IAMs have resulted in the formulation of narratives about future socio-economic conditions that clearly delineate the assumptions for alternative future scenarios \citep{oneill_new_2014, oneill_roads_2017}. These \textit{Shared Socio-economic Pathways} (SSP) \citep{oneill_new_2014} provide the narrative basis for models to determine reduction targets for greenhouse-gas-emissions and to inform global land-use policy, but also for the macro-economic analyses of policy instruments, such as the transition to low-carbon energy \citep{riahi_shared_2017}. 

The main advantage of these narratives are the shared sets of assumptions about alternative, plausible futures that clearly outline the socio-economic environment within which sustainable policy-making takes place. For example, the \textit{scenario matrix architecture} \citep{van_vuuren_new_2014} can be applied to assess the cost and benefits of various policy options (such as carbon pricing) to reduce radiative forcing under each SSP \citep{van_vuuren_new_2014,riahi_shared_2017}. Further harmonization of future socio-economic assumptions has been provided through \textit{Shared policy assumptions} (SPA) \citep{kriegler_new_2014}. Similar to SSP, SPAs harmonize narratives about climate change mitigation policy that are plausible within each SSP, therefore allowing a detailed accounting of costs associated with achieving a certain forcing level under a specific set of mitigation policies under each SSP.

This high level of congruence between the shared scenario assumptions across disciplines provides a promising basis to conduct structured assessments of future biodiversity change \citep{leclere_bending_2020, newbold_global_2015}. Given the institutional support and policy significance of IAM predictions, assessments of future biodiversity change that are conducted under the same assumptions provided by SSP yield tangible results with direct applicability for policy-makers. The structured set-up of SSP and SPA and the corresponding ability to make projections of important environmental drivers allows exploration of the response of biodiversity to climate mitigation policy.

However, IAM that apply SSP to conduct structured scenario predictions are typically developed, maintained and applied by large research institutions. Due to to their sheer complexity, the application of existing IAM outside of their institutional context is generally limited by insufficient resources and expertise (see \Cref{ch1} for a summary of current IAM frameworks used to inform global policy-making). This limitation is further compacted by the focus of IAM on informing global climate change policy and national energy transition strategies. The slow pace at which methods to predict biodiversity outcomes have been integrated in IAM compacts their low implementation rate for predicting biodiversity change.

